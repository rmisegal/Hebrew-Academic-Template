% ============================================================================
% HEBREW ACADEMIC TEMPLATE v5.0 - COMPREHENSIVE EXAMPLE WITH INLINE COMMENTS
% ============================================================================
% Copyright (c) 2025 Dr. Segal Yoram. All rights reserved.
%
% This file demonstrates ALL features of hebrew-academic-template.cls v5.0
% with extensive inline comments explaining best practices and common patterns.
%
% COMPILATION: lualatex → biber → lualatex → lualatex
% ============================================================================

\documentclass{hebrew-academic-template}  % Load the v5.0 template

% ============================================================================
% PREAMBLE - Document Setup
% ============================================================================

% BIBLIOGRAPHY: Must be loaded BEFORE \begin{document}
% The .bib file should have entries with keywords={english} or keywords={hebrew}
\addbibresource{example_references.bib}

% TITLE PAGE INFORMATION: Define metadata for title page
\hebrewtitle{מדריך מקיף לתבנית האקדמית העברית גרסה 5.0}
\englishtitle{Comprehensive Guide to Hebrew Academic Template v5.0}
\hebrewauthor{ד"ר יורם סגל}
\hebrewversion{גרסה \textenglish{5.0}}  % Optional: document version
\date{\textenglish{November 2025}}      % Always wrap dates in \textenglish{}

% ============================================================================
% DOCUMENT BODY
% ============================================================================

\begin{document}

% FRONT MATTER: Title page and table of contents
\maketitle         % Generates title page with all defined metadata
\tableofcontents   % Auto-generated from section commands
\newpage          % Start main content on new page

% ============================================================================
% SECTION 1: BASIC TEXT MIXING
% ============================================================================

% BEST PRACTICE: Always include English translation with \entoc{}
\hebrewsection{יסודות השימוש בתבנית: \entoc{Basic Template Usage}}

% MIXED PARAGRAPH: Demonstrates all text direction commands
זהו מדריך מקיף לשימוש בתבנית האקדמית העברית
\en{hebrew-academic-template.cls}  % English terms MUST use \en{}
גרסה \clsversion.  % Returns "V05-2025-11-09"
התבנית תומכת בכתיבה דו-לשונית עם עברית \en{RTL} ואנגלית \en{LTR},
כולל מספרים כמו \num{123} ו-\num{456.789},  % Numbers MUST use \num{}
אחוזים כמו \percent{95.5},  % Percentages with % symbol
ושנים כמו \hebyear{2025}.  % Years in Hebrew text

% INLINE MATH: Always LTR, no special handling needed
ניתן לשלב נוסחאות מתמטיות כמו $E = mc^2$ או $f(x) = ax^2 + bx + c$
ישירות בתוך הטקסט העברי.

% SUBSECTION: Demonstrates subsection hierarchy
\hebrewsubsection{תכונות עיקריות: \entoc{Key Features}}

% BULLET LIST: Mixed Hebrew/English content
התבנית כוללת את התכונות הבאות:
\begin{itemize}
\item תמיכה מלאה בכיוונים \en{RTL/LTR}  % Always wrap English
\item \num{78} פקודות מיוחדות  % Number of commands
\item \num{8} סביבות מותאמות  % Number of environments
\item \num{24+} חבילות משולבות  % Number of packages
\item תאימות של \percent{100} עם גרסאות קודמות  % Backward compatibility
\item תמיכה בפרקים עם \code{\textbackslash hebrewchapter}  % Inline code
\item טבלאות מתקדמות עם \en{mixed content}  % Technical term
\item קוד \en{Python} עם \en{syntax highlighting}  % Multiple English terms
\end{itemize}

% ============================================================================
% SECTION 2: ADVANCED LANGUAGE SWITCHING
% ============================================================================

\hebrewsection{מעבר מתקדם בין שפות: \entoc{Advanced Language Switching}}

% DEMONSTRATING ALL TEXT DIRECTION COMMANDS
\hebrewsubsection{פקודות כיוון טקסט: \entoc{Text Direction Commands}}

% CODE BLOCK: Shows all available commands
% IMPORTANT: Title can be bilingual, but code must be English-only!
\begin{pythonbox}[פקודות בסיסיות: \en{Basic Commands}]
# Text direction commands (all 15)
\en{English text}           # English in Hebrew context
\heb{Hebrew text}          # Hebrew in English context (rare)
\ilm{inline math/terms}    # Legacy command, use \en{} instead
\num{123.456}             # Numbers with decimal points
\percent{95.5}            # Percentage with % symbol
\hebyear{2025}            # Years in Hebrew text

# Protection and forcing
\ltr{(a+b)}               # Protect brackets from bidi
\LTR{Force LTR}           # Force left-to-right
\RTL{Force RTL}           # Force right-to-left

# Section switching
\startenglish             # Begin English section
\stopenglish              # End English section
\stophebrew               # Alternative ending
\end{pythonbox}

% COMPLEX MIXED EXAMPLE
המחקר של \en{Vaswani et al.} מ-\hebyear{2017} על ארכיטקטורת
\en{Transformer} חולל מהפכה בתחום ה-\en{NLP}. המודל השיג דיוק של
\percent{98.2} במשימת תרגום, עם \num{65} מיליון פרמטרים
ו-\num{8} ראשי \en{attention}.

% ============================================================================
% SECTION 3: TABLES WITH MIXED CONTENT
% ============================================================================

\hebrewsection{טבלאות עם תוכן מעורב: \entoc{Tables with Mixed Content}}

\hebrewsubsection{טבלה בסיסית: \entoc{Basic Table}}

% RTL TABLE: Columns appear right-to-left visually
\begin{hebrewtable}[h]  % [h] = here, [t] = top, [b] = bottom
\caption{השוואת אלגוריתמים: \en{Algorithm Comparison}}  % Bilingual caption
\label{tab:algorithms}  % Label for cross-references
\begin{rtltabular}{|m{3cm}|m{3cm}|m{2cm}|m{2cm}|}  % m{} for vertical centering
\hline
% Headers use \hebheader{} or \enheader{} for proper formatting
\textbf{\hebheader{אלגוריתם}} &
\textbf{\enheader{Algorithm}} &
\textbf{\hebheader{דיוק}} &
\textbf{\enheader{Time (sec)}} \\
\hline
% Data rows: use appropriate cell commands
\hebcell{רגרסיה ליניארית} &  % Hebrew-dominant cell
\encell{Linear Regression} &  % English-only cell
\percent{85.2} &              % Percentage (no cell wrapper needed)
\num{0.1} \\                  % Number (no cell wrapper needed)
\hline
\hebcell{יער אקראי עם \num{100} עצים} &  % Mixed Hebrew cell
\encell{Random Forest (n=100)} &
\percent{92.1} &
\num{5.2} \\
\hline
\hebcell{רשת נוירונים עמוקה \en{DNN}} &  % Hebrew with English acronym
\encell{Deep Neural Network} &
\percent{94.5} &
\num{45.8} \\
\hline
\end{rtltabular}
\end{hebrewtable}

% REFERENCE TO TABLE
כפי שניתן לראות בטבלה \ref{tab:algorithms},
הרשת העמוקה משיגה את הביצועים הטובים ביותר.

% ============================================================================
% SECTION 4: MATHEMATICAL EXPRESSIONS
% ============================================================================

\hebrewsection{ביטויים מתמטיים: \entoc{Mathematical Expressions}}

\hebrewsubsection{משוואות עם טקסט עברי: \entoc{Equations with Hebrew Text}}

% INLINE MATH: Simple formulas in text
הנוסחה הבסיסית $y = mx + b$ מייצגת קו ישר, כאשר $m$ הוא השיפוע.

% DISPLAY MATH: Centered equations
פונקציית העלות למודל רגרסיה:
\begin{equation}
J(\theta) = \frac{1}{2m} \sum_{i=1}^{m}(h_\theta(x^{(i)}) - y^{(i)})^2
\label{eq:cost}  % Label for references
\end{equation}

% HEBREW IN MATH: Using \hebmath{} command
\begin{equation}
\hebmath{מהירות} = \frac{\hebmath{מרחק}}{\hebmath{זמן}}
\end{equation}

% PIECEWISE FUNCTION with Hebrew conditions
\begin{equation}
f(x) = \begin{cases}
  x^2 & \hebmath{אם} \; x > 0 \\
  0 & \hebmath{אם} \; x = 0 \\
  -x^2 & \hebmath{אחרת}
\end{cases}
\end{equation}

% OPTIMIZATION with Hebrew subscripts
\begin{equation}
\theta^* = \argmin_{\theta} L(\theta), \quad
V_{\hebsub{מקסימום}} = \num{100}
\end{equation}

% REFERENCE TO EQUATION
משוואה \eqref{eq:cost} מגדירה את פונקציית העלות.

% ============================================================================
% SECTION 5: CODE BLOCKS
% ============================================================================

\hebrewsection{בלוקי קוד: \entoc{Code Blocks}}

\hebrewsubsection{קוד צף: \entoc{Floating Code}}

% FLOATING CODE: Can move for optimal page layout
\begin{pythonbox}[יישום \en{K-Means}: \en{K-Means Implementation}]
import numpy as np
from sklearn.cluster import KMeans
import matplotlib.pyplot as plt

# Generate sample data
# NOTE: Comments MUST be in English only!
np.random.seed(42)
X = np.random.randn(300, 2)
X[:100] += [2, 2]    # Cluster 1
X[100:200] += [-2, -2]  # Cluster 2

# Apply K-means clustering
kmeans = KMeans(n_clusters=3, random_state=42)
labels = kmeans.fit_predict(X)

# Get cluster centers
centers = kmeans.cluster_centers_
print(f"Converged in {kmeans.n_iter_} iterations")
print(f"Inertia: {kmeans.inertia_:.2f}")

# Visualize results
plt.scatter(X[:, 0], X[:, 1], c=labels, cmap='viridis', alpha=0.6)
plt.scatter(centers[:, 0], centers[:, 1], c='red', marker='X', s=200)
plt.title('K-Means Clustering Results')
plt.show()
\end{pythonbox}

\hebrewsubsection{קוד ארוך לא צף: \entoc{Long Non-Floating Code}}

% NON-FLOATING CODE: Stays in place, can break across pages
% Use pythonbox* (with asterisk) for long code
\begin{pythonbox*}[מודל רשת נוירונים: \en{Neural Network Model}]
import tensorflow as tf
from tensorflow import keras
from tensorflow.keras import layers

# Define model architecture
def create_model(input_shape, num_classes):
    """
    Create a deep neural network model.

    Args:
        input_shape: Shape of input data
        num_classes: Number of output classes

    Returns:
        Compiled Keras model
    """
    model = keras.Sequential([
        layers.Input(shape=input_shape),
        layers.Dense(128, activation='relu'),
        layers.Dropout(0.3),
        layers.Dense(64, activation='relu'),
        layers.Dropout(0.3),
        layers.Dense(32, activation='relu'),
        layers.Dense(num_classes, activation='softmax')
    ])

    model.compile(
        optimizer='adam',
        loss='sparse_categorical_crossentropy',
        metrics=['accuracy']
    )

    return model

# Usage example
model = create_model(input_shape=(784,), num_classes=10)
model.summary()
\end{pythonbox*}

% ============================================================================
% SECTION 6: LISTS AND NUMBERING
% ============================================================================

\hebrewsection{רשימות ומספור: \entoc{Lists and Numbering}}

\hebrewsubsection{רשימות מקוננות: \entoc{Nested Lists}}

% NUMBERED LIST with mixed content
שלבי פיתוח מודל למידת מכונה:
\begin{enumerate}
\item איסוף נתונים: \en{Data Collection}
  \begin{itemize}
  \item מקורות: \en{APIs, Web Scraping, Databases}
  \item גודל: מינימום \num{10,000} דגימות
  \item איכות: לפחות \percent{95} נתונים תקינים
  \end{itemize}

\item עיבוד מקדים: \en{Preprocessing}
  \begin{enumerate}
  \item ניקוי: הסרת \en{outliers}
  \item נרמול: \en{StandardScaler} או \en{MinMaxScaler}
  \item קידוד: \en{One-hot encoding} למשתנים קטגוריים
  \end{enumerate}

\item בחירת מודל: \en{Model Selection}
  \begin{itemize}
  \item פשוטים: \en{Linear/Logistic Regression}
  \item מורכבים: \en{Random Forest, XGBoost}
  \item עמוקים: \en{Neural Networks, Transformers}
  \end{itemize}
\end{enumerate}

% ============================================================================
% SECTION 7: FIGURES
% ============================================================================

\hebrewsection{איורים ותמונות: \entoc{Figures and Images}}

% FIGURE ENVIRONMENT FORM (recommended for v5.0)
\begin{hebrewfigure}[htbp]  % h=here, t=top, b=bottom, p=page
  \centering
  \includegraphics[width=0.7\textwidth]{example_plot.png}
  \caption{תוצאות הניסוי בשנת \hebyear{2025}: \en{Experimental Results 2025}}
  \label{fig:results}
\end{hebrewfigure}

% REFERENCE TO FIGURE
איור \ref{fig:results} מציג את התוצאות המרכזיות.

% ============================================================================
% SECTION 8: CHAPTERS (FOR BOOKS)
% ============================================================================

% CHAPTER: Only for book-length documents
\hebrewchapter{למידה עמוקה}  % No \entoc{} needed for chapters

% Sections within chapters work normally
\hebrewsection{רשתות נוירונים: \entoc{Neural Networks}}

מבנה בסיסי של רשת:
\begin{itemize}
\Hitem{שכבת קלט: \en{Input Layer}}  % \Hitem for Hebrew items
\Hitem{שכבות נסתרות: \en{Hidden Layers}}
\Hitem{שכבת פלט: \en{Output Layer}}
\end{itemize}

% ============================================================================
% SECTION 9: ENGLISH SECTIONS
% ============================================================================

\englishsection{Pure English Section}

% ENGLISH CONTENT: Must be wrapped properly
\startenglish  % Begin English LTR environment

This entire section is in English with left-to-right alignment.
All text here flows naturally in LTR direction.

Key points:
\begin{itemize}
\item Everything is LTR aligned
\item No Hebrew text should appear here
\item Mathematical expressions work: $f(x) = x^2$
\item Numbers don't need special handling: 123, 456
\end{itemize}

\stopenglish  % Return to Hebrew RTL

% ============================================================================
% SECTION 10: FOOTNOTES
% ============================================================================

\hebrewsection{הערות שוליים: \entoc{Footnotes}}

% FOOTNOTE: No space before \footnote{}
טקסט עם הערת שוליים\footnote{זוהי הערת שוליים עם מונח \en{English term}
ומספר \num{42}.} ממשיך כאן.

% MULTIPLE FOOTNOTES
מונחים טכניים\footnote{%
  \en{Machine Learning} - למידת מכונה, תחום המאפשר למחשבים ללמוד מנתונים.%
} חשובים מאוד\footnote{%
  ראו פרק \ref{fig:results} לפרטים נוספים.%
} במחקר מודרני.

% ============================================================================
% SECTION 11: CITATIONS AND BIBLIOGRAPHY
% ============================================================================

\hebrewsection{ציטוטים ומקורות: \entoc{Citations and References}}

% CITATIONS: Various styles
מחקר בודד \cite{mikolov2013}.
מחקרים מרובים \cite{kingma2013,goodfellow2014}.
עם עמודים \cite[עמ' 45-47]{galton1886}.
מחקר של \en{Smith et al.} \cite{mikolov2013} מ-\hebyear{2013}.

% ============================================================================
% SECTION 12: SPECIAL CHARACTERS AND SYMBOLS
% ============================================================================

\hebrewsection{תווים מיוחדים: \entoc{Special Characters}}

% SPECIAL SYMBOLS (v5.0)
\begin{itemize}
\item סימן אזהרה: \warningsymbol{} (משולש שחור)
\item סימן ביקורת: \checksymbol{} (וי)
\item ריבוע R: \Rsquared{} או \Rtwo{}
\item חץ בהקשר RTL: \rarrow{} (מצביע שמאלה)
\end{itemize}

% PATH WITH HYPHENS
הקובץ נמצא ב: \enpath{/usr/local/bin/my-script.sh}

% ============================================================================
% BIBLIOGRAPHY
% ============================================================================

\newpage  % Always start bibliography on new page

% SEPARATE BIBLIOGRAPHIES (optional)
% \printhebrewbibliography   % Hebrew sources only
\printenglishbibliography     % English sources only

% OR SINGLE BIBLIOGRAPHY
% \printbibliography          % All sources together

\end{document}

% ============================================================================
% END OF TEMPLATE EXPLANATION
% ============================================================================
% This template demonstrates all 78 commands and 8 environments
% available in hebrew-academic-template.cls v5.0
%
% Key takeaways:
% 1. Always wrap English with \en{}
% 2. Always wrap numbers with \num{}, \percent{}, or \hebyear{}
% 3. Use bilingual section headings with \entoc{}
% 4. Apply appropriate table cell commands
% 5. Keep code blocks English-only
% 6. Compile with: lualatex → biber → lualatex → lualatex
%
% For more information, see:
% - README.md (overview)
% - USAGE_GUIDE.md (complete command reference)
% - MIXED_CONTENT_GUIDE.md (RTL/LTR rules)
% - MIGRATION_GUIDE.md (upgrading from older versions)
% ============================================================================