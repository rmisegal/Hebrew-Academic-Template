% ================================================
% BIBLIOGRAPHY EXAMPLE - Hebrew Academic Template v5.0
% ================================================
% This example focuses on bibliography features in Hebrew RTL documents
% Compiles to approximately 5-6 pages
% Copyright (c) 2025 Dr. Segal Yoram. All rights reserved.

\documentclass{hebrew-academic-template}

% Add bibliography file
\addbibresource{advanced_references.bib}

% Title page information
\hebrewtitle{דוגמת ביבליוגרפיה - תבנית אקדמית עברית}
\englishtitle{Bibliography Example - Hebrew Academic Template v5.0}
\hebrewauthor{ד"ר סגל יורם}
\date{\textenglish{November 2025}}

\begin{document}

\maketitle
\tableofcontents
\newpage

% ==================== INTRODUCTION ====================

\hebrewsection{מבוא לביבליוגרפיה: \entoc{Introduction to Bibliography}}

מסמך זה מדגים שימוש מקיף במערכת הביבליוגרפיה של התבנית האקדמית העברית.
המערכת תומכת בציטוטים בעברית ובאנגלית, עם הפרדה אוטומטית בין מקורות בשתי השפות.

% ==================== BASIC CITATIONS ====================

\hebrewsection{ציטוטים בסיסיים: \entoc{Basic Citations}}

\hebrewsubsection{ציטוט בודד: \entoc{Single Citation}}

המאמר הפורץ דרך על ארכיטקטורת \en{Transformer} \cite{vaswani2017attention} שינה את תחום עיבוד השפה הטבעית.
מודל \en{BERT} \cite{devlin2018bert} הרחיב את הגישה הזו לכיוון דו-כיווני.

\hebrewsubsection{ציטוטים מרובים: \entoc{Multiple Citations}}

מחקרים רבים בתחום \cite{vaswani2017attention,devlin2018bert,brown2020language} מראים את היעילות של גישות מבוססות \en{attention}.
עבודות נוספות \cite{radford2019language,bert_paper_2018,gpt3_paper_2020} המשיכו לפתח את הטכנולוגיה.

\hebrewsubsection{ציטוטים עם עמודים: \entoc{Citations with Pages}}

כפי שמוצג במחקר \cite[עמ' 15]{vaswani2017attention}, מנגנון הקשב מאפשר למודל להתמקד בחלקים רלוונטיים של הקלט.
במאמר אחר \cite[פרק 3]{devlin2018bert} מוסבר כיצד \en{BERT} משתמש במיסוך אקראי.
עבודות בעברית \cite[עמ' 45-67]{hebrew_nlp_2023} מציגות אתגרים ייחודיים.

% ==================== HEBREW CITATIONS ====================

\hebrewsection{ציטוטים בעברית: \entoc{Hebrew Citations}}

\hebrewsubsection{מקורות עבריים: \entoc{Hebrew Sources}}

מחקרים בשפה העברית \cite{hebrew_nlp_2023,hebrew_linguistics_2022,hebrew_computational_2021} מתמקדים באתגרים הייחודיים של עיבוד עברית.
הבעיות כוללות כתיב חסר וכתיב מלא, ניקוד, וכיוון הכתיבה מימין לשמאל.

הספרות העברית בתחום \cite{hebrew_linguistics_2022} מצביעה על הצורך בגישות מיוחדות.
כפי שמוסבר ב-\cite[פרק 2]{hebrew_computational_2021}, המורפולוגיה העשירה של העברית דורשת טיפול מיוחד.

\hebrewsubsection{ציטוטים מעורבים: \entoc{Mixed Citations}}

שילוב של מקורות עבריים ואנגליים \cite{hebrew_nlp_2023,vaswani2017attention,hebrew_linguistics_2022,devlin2018bert} מאפשר הבנה מקיפה.
המחקר המודרני \cite{brown2020language,hebrew_computational_2021,gpt3_paper_2020} מדגים את החיבור בין גישות גלובליות ומקומיות.

% ==================== CITATION STYLES ====================

\hebrewsection{סגנונות ציטוט: \entoc{Citation Styles}}

\hebrewsubsection{ציטוט בתוך משפט: \entoc{In-Text Citations}}

המחקר של \cite{vaswani2017attention} הציג את מנגנון ה-\en{self-attention}.
בעבודתם, \cite{devlin2018bert} פיתחו את רעיון ה-\en{bidirectional} encoding.
כפי שהראו \cite{brown2020language}, מודלים גדולים משיגים תוצאות מרשימות.

\hebrewsubsection{ציטוט בסוגריים: \entoc{Parenthetical Citations}}

מנגנון הקשב הוא יעיל במיוחד (\cite{vaswani2017attention}).
גישות דו-כיווניות הוכחו כמוצלחות (\cite{devlin2018bert}).
מודלי שפה גדולים מציגים יכולות מרשימות (\cite{brown2020language,gpt3_paper_2020}).

\hebrewsubsection{ציטוטים עם הערות: \entoc{Citations with Notes}}

המחקר \cite[ראה במיוחד עמ' 5]{vaswani2017attention} מדגיש את החשיבות של \en{positional encoding}.
כפי שמוצג ב-\cite[איור 2]{devlin2018bert}, הארכיטקטורה כוללת שכבות מרובות.
ההשוואה ב-\cite[טבלה 3.1]{hebrew_nlp_2023} מראה את הביצועים השונים.

% ==================== ADVANCED BIBLIOGRAPHY ====================

\hebrewsection{ביבליוגרפיה מתקדמת: \entoc{Advanced Bibliography}}

\hebrewsubsection{סוגי פרסומים: \entoc{Publication Types}}

\paragraph{מאמרים בכנסים:}
מאמרי כנסים חשובים \cite{vaswani2017attention,devlin2018bert} מציגים חידושים משמעותיים.

\paragraph{מאמרי כתב עת:}
פרסומים בכתבי עת מובילים \cite{bert_paper_2018,radford2019language} עוברים ביקורת עמיתים קפדנית.

\paragraph{ספרים ופרקי ספרים:}
ספרים מקיפים \cite{deep_learning_book_2021,nlp_survey_2022} מספקים סקירה רחבה.

\paragraph{דוחות טכניים:}
דוחות טכניים \cite{gpt3_paper_2020} מתארים מערכות מורכבות בפירוט.

\hebrewsubsection{קבוצות ציטוטים: \entoc{Citation Groups}}

\paragraph{מודלי שפה:}
המחקרים העיקריים במודלי שפה \cite{radford2019language,brown2020language,gpt3_paper_2020} מדגימים התקדמות מהירה.

\paragraph{עיבוד עברית:}
עבודות בעיבוד עברית \cite{hebrew_nlp_2023,hebrew_linguistics_2022,hebrew_computational_2021} מתמודדות עם אתגרים ייחודיים.

\paragraph{ארכיטקטורות רשת:}
פיתוחי הארכיטקטורה \cite{vaswani2017attention,devlin2018bert,bert_paper_2018} הביאו לשיפורים דרמטיים.

% ==================== CITATION MANAGEMENT ====================

\hebrewsection{ניהול ציטוטים: \entoc{Citation Management}}

\hebrewsubsection{ארגון המקורות: \entoc{Source Organization}}

המערכת מאפשרת:
\begin{enumerate}
    \item הפרדה אוטומטית בין מקורות עבריים ואנגליים
    \item מיון אלפביתי בכל קטגוריה
    \item תמיכה במגוון סוגי פרסומים
    \item עיצוב מותאם לכל שפה
\end{enumerate}

\hebrewsubsection{שימוש ב-keywords: \entoc{Using Keywords}}

כל מקור בקובץ ה-\texttt{.bib} צריך לכלול:
\begin{itemize}
    \item \texttt{keywords=\{hebrew\}} למקורות בעברית
    \item \texttt{keywords=\{english\}} למקורות באנגלית
    \item שדה זה קובע באיזה חלק יופיע המקור
\end{itemize}

% ==================== CROSS-REFERENCING ====================

\hebrewsection{הפניות צולבות: \entoc{Cross-References}}

\hebrewsubsection{הפניות למקורות קשורים: \entoc{Related Sources}}

המחקר של \cite{vaswani2017attention} היווה בסיס לעבודות רבות.
למשל, \cite{devlin2018bert} הרחיב את הרעיונות הללו, ו-\cite{brown2020language} לקח אותם לכיוון חדש.

בהקשר העברי, \cite{hebrew_nlp_2023} מתבסס על \cite{hebrew_linguistics_2022} ומרחיב את \cite{hebrew_computational_2021}.

\hebrewsubsection{ציטוטים משווים: \entoc{Comparative Citations}}

השוואה בין גישות שונות:
\begin{itemize}
    \item גישת \cite{vaswani2017attention}: מתמקדת ב-\en{attention}
    \item גישת \cite{devlin2018bert}: מוסיפה \en{bidirectionality}
    \item גישת \cite{brown2020language}: מדגישה \en{scale}
\end{itemize}

% ==================== SPECIAL CASES ====================

\hebrewsection{מקרים מיוחדים: \entoc{Special Cases}}

\hebrewsubsection{ציטוטים ארוכים: \entoc{Long Citations}}

לפעמים נדרש לצטט מקורות רבים בנושא מסוים \cite{vaswani2017attention,devlin2018bert,radford2019language,brown2020language,bert_paper_2018,gpt3_paper_2020,hebrew_nlp_2023,hebrew_linguistics_2022,hebrew_computational_2021,deep_learning_book_2021,nlp_survey_2022}.

\hebrewsubsection{ציטוטים בטבלאות: \entoc{Citations in Tables}}

\begin{hebrewtable}[h]
\caption{השוואת מחקרים: \entoc{Research Comparison}}
\begin{rtltabular}{|c|c|c|}
\hline
\mixedcell{\textbf{מחקר / \en{Study}}} & \mixedcell{\textbf{שנה / \en{Year}}} & \mixedcell{\textbf{תרומה / \en{Contribution}}} \\
\hline
\cite{vaswani2017attention} & \hebyear{2017} & \en{Self-attention} \\
\hline
\cite{devlin2018bert} & \hebyear{2018} & \en{Bidirectional} \\
\hline
\cite{brown2020language} & \hebyear{2020} & \en{GPT-3 scale} \\
\hline
\cite{hebrew_nlp_2023} & \hebyear{2023} & \mixedcell{עיבוד עברית} \\
\hline
\end{rtltabular}
\end{hebrewtable}

\hebrewsubsection{ציטוטים בהערות שוליים: \entoc{Citations in Footnotes}}

טקסט עם הערה וציטוט\footnote{ראה את המחקר של \cite{vaswani2017attention} לפרטים נוספים על מנגנון הקשב.}.
הערה נוספת עם מספר ציטוטים\footnote{מחקרים רבים \cite{devlin2018bert,brown2020language} תומכים בגישה זו.}.

% ==================== ENGLISH SECTION ====================

\englishsection{English Bibliography Section}

\startenglish
This section demonstrates citations in pure English text.
The transformer architecture \cite{vaswani2017attention} revolutionized NLP.
BERT \cite{devlin2018bert} and GPT \cite{radford2019language,brown2020language} built upon these foundations.

Recent surveys \cite{nlp_survey_2022,deep_learning_book_2021} provide comprehensive overviews of the field.
Hebrew-specific research \cite{hebrew_nlp_2023,hebrew_linguistics_2022} addresses unique challenges.
\stopenglish

% ==================== SUMMARY ====================

\hebrewsection{סיכום: \entoc{Summary}}

מסמך זה הדגים:

\begin{itemize}
    \item ציטוטים בודדים ומרובים
    \item ציטוטים עם הפניות לעמודים ופרקים
    \item הפרדה בין מקורות עבריים ואנגליים
    \item סגנונות ציטוט שונים (בתוך משפט, בסוגריים)
    \item ציטוטים בטבלאות, הערות שוליים, ואיורים
    \item ניהול וארגון ביבליוגרפיה
    \item שימוש ב-\texttt{keywords} להפרדת שפות
\end{itemize}

המערכת מספקת גמישות מלאה בניהול מקורות ביבליוגרפיים במסמכים אקדמיים דו-לשוניים.

% ==================== BIBLIOGRAPHY ====================

\newpage
% Demonstrate the split bibliography
\printhebrewbibliography
\printenglishbibliography

\end{document}