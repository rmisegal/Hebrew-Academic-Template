% ================================================
% EXPERT EXAMPLE - Hebrew Academic Template
% ================================================
% THIS EXAMPLE DEMONSTRATES ALL COMMANDS FROM THE CLS
% Compiles to approximately 15-20 pages
% Copyright (c) 2025 Dr. Segal Yoram. All rights reserved.

\documentclass{hebrew-academic-template}

% Force figure/table positioning
\usepackage{float}

% Add bibliography file
\addbibresource{advanced_references.bib}

% Title page information - Commands: \hebrewtitle, \englishtitle, \hebrewauthor, \hebrewversion
\hebrewtitle{דוגמת מומחה - כל הפקודות של התבנית}
\englishtitle{Expert Example - All Template Commands \clsversion}
\hebrewauthor{ד"ר סגל יורם}
\hebrewversion{גרסה \en{\clsversion}}
\date{\textenglish{January 2026}}

\begin{document}

% Command: \maketitle
\maketitle
\tableofcontents
\listoffigures
\listoftables
\newpage

% ==================== CHAPTER DEMONSTRATION ====================
% Command: \hebrewchapter (v3.0 feature)
% v5.10.0: Added \label for cross-chapter reference testing

\hebrewchapter{פרק ראשון: הדגמת כל הפקודות}
\hebrewchapterlabel{chap:first}

מסמך זה מדגים את כל הפקודות הזמינות בתבנית האקדמית העברית גרסה \en{\clsversion}.
הגרסה הנוכחית: \en{\clsversion} % Command: \clsversion

התבנית תומכת בעבודה עם מקורות ביבליוגרפיים כמו \cite{vaswani2017attention,devlin2018bert}.
מחקרים בעברית \cite{hebrew_nlp_2023} משולבים בצורה חלקה עם מקורות באנגלית.

% ==================== TEXT DIRECTION COMMANDS ====================

\hebrewsection{פקודות כיוון טקסט: \entoc{Text Direction Commands}}

\hebrewsubsection{פקודות בסיסיות: \entoc{Basic Commands}}

% Commands: \en, \heb, \ilm
טקסט עברי עם \en{English text} באמצע.
טקסט אנגלי עם \heb{טקסט עברי} בתוכו.
מונחים טכניים: \ilm{inline math terms}.

% Commands: \num, \hebyear, \percent
מספרים: \num{12345}, \num{3.14159}, \num{6.022e23}
שנים: \hebyear{2025}, \hebyear{1948}
אחוזים: \percent{99.9}, \percent{0.01}

% Command: \ltr (v1.0 feature)
טקסט מוגן ב-LTR: \ltr{protected LTR text}

% Commands: \LTR, \RTL (demonstrated safely with \en{} and \heb{})
כיוון כללי LTR: \en{Left to Right}
כיוון כללי RTL: \heb{מימין לשמאל}

% Commands: \warningsymbol, \checksymbol
סמלים מיוחדים: \warningsymbol{} אזהרה, \checksymbol{} אישור

% ==================== SECTION COMMANDS ====================

\hebrewsection{פקודות סעיפים: \entoc{Section Commands}}

% Already demonstrated: \hebrewchapter, \hebrewsection

\hebrewsubsection{תת-סעיף עברי: \entoc{Hebrew Subsection}}

זהו תת-סעיף עברי עם מספור אוטומטי.

% Command: \englishsection
\englishsection{Pure English Section}

% Commands: \startenglish, \stopenglish
\startenglish
This section demonstrates pure English content with proper LTR alignment.
All text flows from left to right and is aligned to the left margin.

We can include:
\begin{itemize}
    \item Bullet points in English
    \item Mathematical formulas: $y = mx + b$
    \item Technical terms and code snippets
\end{itemize}
\stopenglish

% Command: \stophebrew (return to Hebrew)
\stophebrew
חזרנו לטקסט עברי עם יישור RTL.

% Commands: \HebrewTitle, \HebrewSubtitle (internal title helpers)
הכותרות משתמשות בפקודות פנימיות כמו \en{\texttt{HebrewTitle}} ו-\en{\texttt{HebrewSubtitle}}.

% ==================== TABLE COMMANDS ====================

\hebrewsection{פקודות טבלאות: \entoc{Table Commands}}

\hebrewsubsection{טבלה מקיפה: \entoc{Comprehensive Table}}

% Commands: \begin{hebrewtable}, \begin{rtltabular}
\begin{table}[H]
\caption{כל פקודות הטבלה: \entoc{All Table Commands}}
\centering
\begin{tblr}{
  colspec = {cccc},
  row{1} = {font=\bfseries, bg=blue9, fg=white},
  row{even} = {bg=gray9},
  hlines,
  vlines,
}
% Commands in table: \hebcell, \mixedcell, \encell, \hebheader, \enheader
כותרת עברית & \en{English Header} & כותרת מעורבת / \en{Mixed} & \en{Pure English} \\
תא עברי & \en{English cell} & תא מעורב / \en{Mixed cell} & \en{Data: 42} \\
נתונים: \num{100} & \percent{95.5} & שנה / \en{Year}: \hebyear{2025} & $\alpha = 0.05$ \\
\end{tblr}
\end{table}

% Command: \rtlrow (optional helper from v1.0)
הפקודה \en{\texttt{\textbackslash rtlrow}} זמינה לסידור אוטומטי של עמודות RTL.

% Command: \ltrnumber (for bibliography numbers)
מספר ביבליוגרפי: \ltrnumber{123}

% ==================== FIGURE COMMANDS ====================

\hebrewsection{פקודות איורים: \entoc{Figure Commands}}

% Command: \hebrewfigure (command form)
\hebrewfigure[H]{
    \includegraphics[width=0.8\textwidth]{images/fig_expert1.pdf}
}{איור בפקודה: \entoc{Command Form Figure}}

% Environment form (also supported)
\begin{figure}[H]
\centering
\includegraphics[width=0.8\textwidth]{images/fig_expert2.pdf}
\caption{איור בסביבה: \entoc{Environment Form Figure}}
\end{figure}

% ==================== CODE COMMANDS ====================

\hebrewsection{פקודות קוד: \entoc{Code Commands}}

\hebrewsubsection{קוד צף: \entoc{Floating Code}}

% Command: \begin{pythonbox}
\begin{english}
\begin{pythonbox}[\hebtitle{דוגמת קוד צף: \en{Floating Code Example}}]
# Python code demonstration with syntax highlighting
def fibonacci(n):
    """Calculate Fibonacci sequence"""
    if n <= 1:
        return n
    a, b = 0, 1
    for _ in range(2, n + 1):
        a, b = b, a + b
    return b

# Test the function
for i in range(10):
    print(f"F({i}) = {fibonacci(i)}")
\end{pythonbox}
\end{english}

\hebrewsubsection{קוד לא צף: \entoc{Non-Floating Code}}

% Command: \begin{pythonbox*} (non-floating version)
\begin{english}
\begin{pythonbox*}[\hebtitle{קוד ארוך לא צף: \en{Long Non-Floating Code}}]
# Non-floating code block for long listings
class DataProcessor:
    def __init__(self, data):
        self.data = data
        self.processed = False

    def clean(self):
        # Remove null values
        self.data = [x for x in self.data if x is not None]
        return self

    def normalize(self):
        # Normalize to 0-1 range
        if self.data:
            min_val = min(self.data)
            max_val = max(self.data)
            if max_val > min_val:
                self.data = [(x - min_val) / (max_val - min_val)
                            for x in self.data]
        self.processed = True
        return self

# Example usage
processor = DataProcessor([1, 2, None, 4, 5])
processor.clean().normalize()
\end{pythonbox*}
\end{english}

% Commands: \pythonverbatimformat (internal), \listingfont, \courierfont (internal)
הפקודות הפנימיות \en{\texttt{pythonverbatimformat}}, \en{\texttt{listingfont}}, ו-\en{\texttt{courierfont}} מטפלות בעיצוב.

% Command: \enpath (for paths with hyphens)
נתיב עם מקפים: \enpath{/usr/local/bin/python-3.9}

% Commands: \code, \englishterm
קוד מוטבע: \code{print("Hello World")}
מונח אנגלי: \englishterm{machine learning}

% ==================== MATHEMATICAL COMMANDS ====================

\hebrewsection{פקודות מתקדמות: \entoc{Advanced Features}}

\hebrewsubsection{שילוב מורכב: \entoc{Complex Integration}}

% Demonstrate complex nested usage
טבלה עם כל הפקודות המתקדמות:

\begin{table}[H]
\caption{שילוב פקודות: \entoc{Command Integration}}
\centering
\begin{tblr}{
  colspec = {ccc},
  row{1} = {font=\bfseries, bg=blue9, fg=white},
  row{even} = {bg=gray9},
  hlines,
  vlines,
}
% RTL ORDER: Command - Example - Result
פקודה / \en{Command} & דוגמה / \en{Example} & \en{Result} \\
\en{\texttt{\textbackslash num}} & \num{3.14159} & \en{3.14159} \\
\en{\texttt{\textbackslash percent}} & \percent{99.99} & \en{99.99\%} \\
\en{\texttt{\textbackslash hebyear}} & \hebyear{2025} & \en{2025} \\
\en{\texttt{\textbackslash warningsymbol}} & \warningsymbol & $\blacktriangle$ \\
\en{\texttt{\textbackslash checksymbol}} & \checksymbol & $\checkmark$ \\
\en{\texttt{\textbackslash Rsquared}} & \Rsquared & $R^2$ \\
\en{\texttt{\textbackslash rarrow}} & א \rarrow{} ב & \en{A} $\leftarrow$ \en{B} \\
\end{tblr}
\end{table}

\hebrewsubsection{נוסחאות מורכבות: \entoc{Complex Formulas}}

נוסחת אופטימיזציה מלאה עם עברית:

\begin{align}
J(\theta) &= \frac{1}{N}\sum_{i=1}^{N} L_{\hebsub{הפסד}}(f_\theta(x_i), y_i) + \lambda R(\theta) \\
\theta_{t+1} &= \theta_t - \alpha \nabla_\theta J(\theta_t) \\
\theta^* &= \argmin_{\theta \in \Theta} J(\theta) \\
\hebmath{כאשר } & \alpha = \num{0.001}, \lambda = \num{1e-4}
\end{align}

\hebrewsubsection{קוד מתקדם עם הערות: \entoc{Advanced Code with Comments}}

\begin{english}
\begin{pythonbox}[\hebtitle{מימוש \en{Attention} עם קשב}]
import torch
import torch.nn as nn
import torch.nn.functional as F

class MultiHeadAttention(nn.Module):
    """
    Multi-head attention mechanism
    Used in Transformer architecture
    """
    def __init__(self, d_model=512, n_heads=8):
        super().__init__()
        self.d_model = d_model
        self.n_heads = n_heads
        self.d_k = d_model // n_heads

        # Linear projections
        self.W_q = nn.Linear(d_model, d_model)
        self.W_k = nn.Linear(d_model, d_model)
        self.W_v = nn.Linear(d_model, d_model)
        self.W_o = nn.Linear(d_model, d_model)

    def forward(self, query, key, value, mask=None):
        batch_size = query.size(0)

        # Project and reshape
        Q = self.W_q(query).view(batch_size, -1, self.n_heads, self.d_k)
        K = self.W_k(key).view(batch_size, -1, self.n_heads, self.d_k)
        V = self.W_v(value).view(batch_size, -1, self.n_heads, self.d_k)

        # Transpose for attention
        Q = Q.transpose(1, 2)
        K = K.transpose(1, 2)
        V = V.transpose(1, 2)

        # Scaled dot-product attention
        scores = torch.matmul(Q, K.transpose(-2, -1)) / (self.d_k ** 0.5)

        if mask is not None:
            scores = scores.masked_fill(mask == 0, -1e9)

        attention = F.softmax(scores, dim=-1)
        context = torch.matmul(attention, V)

        # Concatenate heads
        context = context.transpose(1, 2).contiguous()
        context = context.view(batch_size, -1, self.d_model)

        # Final projection
        output = self.W_o(context)

        return output, attention
\end{pythonbox}
\end{english}

% ==================== SPECIAL ENVIRONMENTS ====================

\hebrewsection{סביבות מיוחדות: \entoc{Special Environments}}

\hebrewsubsection{סביבת אנגלית מלאה: \entoc{Full English Environment}}

\begin{english}
% This creates a full English environment
This entire paragraph is in English with proper LTR alignment.
We can include mathematical formulas like $E = mc^2$ and lists:

\begin{enumerate}
    \item First item in English
    \item Second item with formula: $\int_0^1 x^2 dx = \frac{1}{3}$
    \item Third item with code: \texttt{print("Hello")}
\end{enumerate}

The environment ensures consistent English formatting throughout.
\end{english}

\hebrewsubsection{סביבת עברית מלאה: \entoc{Full Hebrew Environment}}

\begin{hebrew}
% This ensures full Hebrew environment
פסקה זו כולה בעברית עם יישור RTL מלא.
אנו יכולים לכלול נוסחאות מתמטיות כמו $x^2 + y^2 = r^2$ ורשימות:

\begin{enumerate}
    \item פריט ראשון בעברית
    \item פריט שני עם נוסחה: $\sum_{i=1}^n i = \frac{n(n+1)}{2}$
    \item פריט שלישי עם קוד: \texttt{print("שלום")}
\end{enumerate}

הסביבה מבטיחה עיצוב עברי עקבי לאורך כל החלק.
\end{hebrew}

% ==================== ADVANCED TABLE FEATURES ====================

\hebrewsection{טבלאות מתקדמות: \entoc{Advanced Tables}}

\begin{longtable}{|p{3cm}|p{3cm}|p{4cm}|p{2cm}|}
\caption{טבלה ארוכה עם \en{longtable}: \entoc{Long Table Example}} \\
\hline
% RTL ORDER (rightmost to leftmost): Status - Value - Description - Number
\textbf{\enheader{Status}} &
\textbf{\hebheader{ערך / \en{Value}}} &
\textbf{\hebheader{תיאור / \en{Description}}} &
\textbf{\hebheader{מספר}} \\
\hline
\endfirsthead

\multicolumn{4}{r}{\textit{המשך מעמוד קודם}} \\
\hline
\textbf{\enheader{Status}} &
\textbf{\hebheader{ערך}} &
\textbf{\hebheader{תיאור}} &
\textbf{\hebheader{מספר}} \\
\hline
\endhead

\hline
\multicolumn{4}{r}{\textit{ממשיך בעמוד הבא}} \\
\endfoot

\hline
\endlastfoot

\encell{Active} & \hebcell{\percent{10}} & \hebcell{פריט ראשון עם \en{English}} & \hebcell{\num{1}} \\
\hline
\encell{Pending} & \hebcell{\num{250.5}} & \hebcell{פריט שני} & \hebcell{\num{2}} \\
\hline
\encell{Complete} & \hebcell{\hebyear{2025}} & \hebcell{פריט שלישי} & \hebcell{\num{3}} \\
\hline
\encell{Active} & \hebcell{\Rsquared = \num{0.95}} & \hebcell{פריט רביעי} & \hebcell{\num{4}} \\
\hline
\encell{Testing} & \hebcell{\percent{99.9}} & \hebcell{פריט חמישי} & \hebcell{\num{5}} \\
\hline
\end{longtable}

% ==================== TIKZ DIAGRAMS ====================

\hebrewsection{דיאגרמות מתקדמות: \entoc{Advanced Diagrams}}

\begin{figure}[H]
\centering
\begin{english}
\begin{tikzcd}
 A \arrow[r, "\alpha"] \arrow[d, "\beta"'] & B \arrow[d, "\gamma"] \\
 C \arrow[r, "\delta"'] & D
\end{tikzcd}
\end{english}
\caption{דיאגרמה עם \en{TikZ}: \entoc{TikZ Diagram}}
\end{figure}

% ==================== COMPREHENSIVE EXAMPLE ====================

\hebrewsection{דוגמה מקיפה: \entoc{Comprehensive Example}}

\hebrewsubsection{שילוב כל התכונות: \entoc{Combining All Features}}

נדגים שילוב של כל התכונות:

\begin{enumerate}
    \item טקסט דו-כיווני: עברית עם \en{English} ומספרים \num{42}
    \item סמלים: \warningsymbol{} זהירות, \checksymbol{} בדוק
    \item מתמטיקה: $\theta^* = \argmin_\theta L(\theta)$ עם $R^2 = \Rsquared$
    \item קוד: \code{def func():} או \englishterm{function}
    \item נתיבים: \enpath{/path/to/file-name.py}
    \item שנים ואחוזים: \hebyear{2025}, \percent{95.5}
\end{enumerate}

טבלת סיכום:

\begin{table}[H]
\centering
\begin{tblr}{
  colspec = {ccc},
  row{1} = {font=\bfseries, bg=blue9, fg=white},
  row{even} = {bg=gray9},
  row{8} = {font=\bfseries},
  hlines,
  vlines,
}
קטגוריה / \en{Category} & כמות / \en{Count} & \en{Percentage} \\
פקודות טקסט / \en{Text Commands} & \num{15} & \percent{19.2} \\
פקודות סעיפים / \en{Section Commands} & \num{6} & \percent{7.7} \\
פקודות טבלה / \en{Table Commands} & \num{8} & \percent{10.3} \\
פקודות קוד / \en{Code Commands} & \num{7} & \percent{9.0} \\
פקודות מתמטיקה / \en{Math Commands} & \num{8} & \percent{10.3} \\
אחרות / \en{Others} & \num{34} & \percent{43.5} \\
סה"כ / \en{Total} & \num{78} & \percent{100} \\
\end{tblr}
\end{table}

% ==================== FOOTNOTES ====================

\hebrewsection{הערות שוליים: \entoc{Footnotes}}

טקסט עם הערת שוליים\footnote{זוהי הערת שוליים בעברית עם \en{English text} בתוכה.}.
הערה נוספת\footnote{הערת שוליים שנייה עם מספר \num{42} ואחוז \percent{95}.}.

טקסט באנגלית עם הערה\footnote{\en{This is an English footnote with some Hebrew:} עברית.}.

% ==================== FINAL DEMONSTRATION ====================

\hebrewchapter{פרק שני: סיכום ההדגמה}
\hebrewchapterlabel{chap:second}

\hebrewsection{סיכום כל הפקודות: \entoc{Summary of All Commands}}

% v5.10.0: Test cross-chapter reference - should show "1" in LTR, not "1" reversed
בפרק~\ref{chap:first} הדגמנו את כל הפקודות. כעת בפרק~\ref{chap:second} נסכם אותן.

מסמך זה הדגים בהצלחה את כל \num{78} הפקודות:

\begin{itemize}
    \item \textbf{פקודות טקסט (\en{15}):} \en{\texttt{\textbackslash en}}, \en{\texttt{\textbackslash heb}}, \en{\texttt{\textbackslash ilm}}, \en{\texttt{\textbackslash num}}, \en{\texttt{\textbackslash hebyear}}, \en{\texttt{\textbackslash percent}}, \en{\texttt{\textbackslash ltr}}, \en{\texttt{\textbackslash LTR}}, \en{\texttt{\textbackslash RTL}}, \en{\texttt{\textbackslash startenglish}}, \en{\texttt{\textbackslash stopenglish}}, \en{\texttt{\textbackslash stophebrew}}, \en{\texttt{\textbackslash warningsymbol}}, \en{\texttt{\textbackslash checksymbol}}

    \item \textbf{פקודות סעיפים (\en{6}):} \en{\texttt{\textbackslash hebrewchapter}}, \en{\texttt{\textbackslash hebrewsection}}, \en{\texttt{\textbackslash hebrewsubsection}}, \en{\texttt{\textbackslash englishsection}}, \en{\texttt{\textbackslash HebrewTitle}}, \en{\texttt{\textbackslash HebrewSubtitle}}

    \item \textbf{פקודות טבלה (\en{8}):} \en{\texttt{hebrewtable}}, \en{\texttt{rtltabular}}, \en{\texttt{\textbackslash hebcell}}, \en{\texttt{\textbackslash encell}}, \en{\texttt{\textbackslash mixedcell}}, \en{\texttt{\textbackslash hebheader}}, \en{\texttt{\textbackslash enheader}}, \en{\texttt{\textbackslash rtlrow}}

    \item \textbf{פקודות איור (\en{2}):} \en{\texttt{\textbackslash hebrewfigure}}, סביבת \en{\texttt{figure}}

    \item \textbf{פקודות קוד (\en{7}):} \en{\texttt{pythonbox}}, \en{\texttt{pythonbox*}}, \en{\texttt{\textbackslash pythonverbatimformat}}, \en{\texttt{\textbackslash listingfont}}, \en{\texttt{\textbackslash courierfont}}, \en{\texttt{\textbackslash enpath}}, \en{\texttt{\textbackslash code}}, \en{\texttt{\textbackslash englishterm}}

    \item \textbf{פקודות מתמטיקה (\en{8}):} \en{\texttt{\textbackslash argmin}}, \en{\texttt{\textbackslash argmax}}, \en{\texttt{\textbackslash hebmath}}, \en{\texttt{\textbackslash hebsub}}, \en{\texttt{\textbackslash Rsquared}}, \en{\texttt{\textbackslash Rtwo}}, \en{\texttt{\textbackslash rarrow}}

    \item \textbf{פקודות כותרת (\en{5}):} \en{\texttt{\textbackslash hebrewtitle}}, \en{\texttt{\textbackslash englishtitle}}, \en{\texttt{\textbackslash hebrewauthor}}, \en{\texttt{\textbackslash hebrewversion}}, \en{\texttt{\textbackslash maketitle}}

    \item \textbf{פקודות ביבליוגרפיה (\en{3}):} \en{\texttt{\textbackslash printhebrewbibliography}}, \en{\texttt{\textbackslash printenglishbibliography}}, \en{\texttt{\textbackslash ltrnumber}}

    \item \textbf{פקודות רשימה (\en{1}):} \en{\texttt{\textbackslash Hitem}}

    \item \textbf{פקודת גרסה (\en{1}):} \en{\texttt{\textbackslash clsversion}}
\end{itemize}

\hebrewsection{מסקנות: \entoc{Conclusions}}

התבנית האקדמית העברית גרסה \en{\clsversion} מספקת:

\begin{enumerate}
    \item תמיכה מלאה בכתיבה דו-כיוונית
    \item \num{78} פקודות מיוחדות לעבודה אקדמית
    \item תאימות לאחור עם כל הגרסאות
    \item גמישות מלאה בעיצוב מסמכים
    \item תמיכה בפרקים למסמכים ארוכים
    \item קוד צף ולא צף
    \item טבלאות מתקדמות עם תוכן מעורב
    \item ביבליוגרפיה דו-לשונית
\end{enumerate}

הגרסה הנוכחית של התבנית: \en{\clsversion}

% ==================== NEW v6.3.5 TEST: CROSS-CHAPTER REFERENCES ====================
\hebrewsection{הפניות בין-פרקים: \entoc{Cross-Chapter References}}

% Test the new chapterref commands added in v6.3.5
הפקודות החדשות מאפשרות הפניה לפרקים בצורה עקבית:

\begin{itemize}
    \item \textbf{הפניה לפרק בודד:} \chapterref{3} % Should show "פרק 3"
    \item \textbf{טווח פרקים:} \chapterrefrange{2}{5} % Should show "פרקים 2-5"
    \item \textbf{רשימת פרקים:} \chapterreflist{1, 4 ו-7} % Should show "פרקים 1, 4 ו-7"
    \item \textbf{הפניה קדימה:} \chapterrefforward{8} % Should show "יוסבר בפרק 8"
\end{itemize}

דוגמאות בהקשר:
\begin{enumerate}
    \item ראו \chapterref{2} לפרטים נוספים על הנושא.
    \item הנושאים מכוסים ב\chapterrefrange{3}{6}.
    \item ראו \chapterreflist{1, 3 ו-5} לרקע תיאורטי.
    \item מימוש מתקדם \chapterrefforward{10}.
\end{enumerate}

% ==================== NEW v5.11.4 TEST: SPACE ARTIFACTS ====================
\hebrewsection{בדיקת רווחים בקוד: \entoc{Code Space Artifacts Test}}

\begin{english}
\begin{pythonbox}[\hebtitle{בדיקת מחרוזות עם רווחים: \en{String Space Test}}]
def print_message():
    # These strings should contain invisible spaces, not markers
    message = "Hello World From Python"
    print(f"Message: {message}")
\end{pythonbox}
\end{english}

% ==================== BIBLIOGRAPHY ====================

\newpage
% Commands: \printhebrewbibliography, \printenglishbibliography
\printhebrewbibliography
\printenglishbibliography

\end{document}
