% ================================================
% IMAGE EXAMPLE - Hebrew Academic Template
% ================================================
% This example focuses on figures and images in Hebrew RTL documents
% Compiles to approximately 4-5 pages
% Copyright (c) 2025 Dr. Segal Yoram. All rights reserved.

\documentclass{hebrew-academic-template}

% Add bibliography file
\addbibresource{example_references.bib}

% Title page information
\hebrewtitle{דוגמת איורים ותמונות - תבנית אקדמית עברית}
\englishtitle{Image and Figure Example - Hebrew Academic Template \clsversion}
\hebrewauthor{ד"ר סגל יורם}
\date{\textenglish{November 2025}}

\begin{document}

\maketitle
\tableofcontents
\listoffigures
\newpage

% ==================== INTRODUCTION ====================

\hebrewsection{מבוא לאיורים: \entoc{Introduction to Figures}}

מסמך זה מדגים שימוש מקיף באיורים ותמונות בתבנית האקדמית העברית.
איורים הם אלמנט חיוני במסמכים אקדמיים ומשמשים להמחשה ויזואלית של רעיונות מורכבים.

% ==================== BASIC FIGURES ====================

\hebrewsection{איורים בסיסיים: \entoc{Basic Figures}}

\hebrewsubsection{איור פשוט: \entoc{Simple Figure}}

% Using standard figure environment
\begin{figure}[h]
\centering
\fbox{\parbox{10cm}{
    \centering
    \vspace{3cm}
    מקום לתמונה\\
    \en{Image Placeholder}\\
    \vspace{3cm}
}}
\caption{איור בסיסי: \entoc{Basic Figure} - דוגמה פשוטה}
\label{fig:basic}
\end{figure}

כפי שניתן לראות באיור \ref{fig:basic}, זהו איור בסיסי עם כיתוב דו-לשוני.

\hebrewsubsection{איור עם פקודת hebrewfigure: \entoc{Figure with hebrewfigure Command}}

% Using \hebrewfigure command
\hebrewfigure[h]{
    \fbox{\parbox{8cm}{
        \centering
        \vspace{2cm}
        איור עם \texttt{hebrewfigure}\\
        \en{Using hebrewfigure command}\\
        \vspace{2cm}
    }}
}{איור עם פקודה: \entoc{Command Figure} - שימוש בפקודת \texttt{hebrewfigure}}

% ==================== POSITIONING OPTIONS ====================

\hebrewsection{אפשרויות מיקום: \entoc{Positioning Options}}

\hebrewsubsection{מיקום בראש העמוד: \entoc{Top Position}}

\begin{figure}[t]
\centering
\fbox{\parbox{12cm}{
    \centering
    \vspace{2cm}
    איור בראש העמוד\\
    \en{Top-positioned figure}\\
    מיקום: \texttt{[t]}\\
    \vspace{2cm}
}}
\caption{איור עם מיקום עליון: \entoc{Top-Positioned Figure} - משתמש ב-\texttt{[t]}}
\label{fig:top}
\end{figure}

איור \ref{fig:top} ממוקם בראש העמוד באמצעות האפשרות \texttt{[t]}.

\hebrewsubsection{מיקום בתחתית העמוד: \entoc{Bottom Position}}

\begin{figure}[b]
\centering
\fbox{\parbox{12cm}{
    \centering
    \vspace{2cm}
    איור בתחתית העמוד\\
    \en{Bottom-positioned figure}\\
    מיקום: \texttt{[b]}\\
    \vspace{2cm}
}}
\caption{איור עם מיקום תחתון: \entoc{Bottom-Positioned Figure} - משתמש ב-\texttt{[b]}}
\label{fig:bottom}
\end{figure}

איור \ref{fig:bottom} ממוקם בתחתית העמוד באמצעות האפשרות \texttt{[b]}.

\hebrewsubsection{מיקום בעמוד נפרד: \entoc{Page Position}}

\begin{figure}[p]
\centering
\fbox{\parbox{14cm}{
    \centering
    \vspace{5cm}
    איור גדול בעמוד נפרד\\
    \en{Large figure on separate page}\\
    מיקום: \texttt{[p]}\\
    \vspace{5cm}
}}
\caption{איור בעמוד נפרד: \entoc{Page Figure} - משתמש ב-\texttt{[p]}}
\label{fig:page}
\end{figure}

איור \ref{fig:page} ממוקם בעמוד נפרד באמצעות האפשרות \texttt{[p]}.

% ==================== MULTIPLE FIGURES ====================

\hebrewsection{איורים מרובים: \entoc{Multiple Figures}}

\hebrewsubsection{איורים זה לצד זה: \entoc{Side-by-Side Figures}}

\begin{figure}[h]
\centering
\begin{tabular}{cc}
\fbox{\parbox{6cm}{
    \centering
    \vspace{2cm}
    איור ימני\\
    \en{Right Figure}\\
    \vspace{2cm}
}} &
\fbox{\parbox{6cm}{
    \centering
    \vspace{2cm}
    איור שמאלי\\
    \en{Left Figure}\\
    \vspace{2cm}
}}
\end{tabular}
\caption{איורים זה לצד זה: \entoc{Side-by-Side Figures}}
\label{fig:sidebyside}
\end{figure}

איור \ref{fig:sidebyside} מדגים שני איורים זה לצד זה.

\hebrewsubsection{רשת איורים: \entoc{Figure Grid}}

\begin{figure}[h]
\centering
\begin{tabular}{cc}
\fbox{\parbox{5cm}{\centering\vspace{2cm}איור 1\\[2cm]}} &
\fbox{\parbox{5cm}{\centering\vspace{2cm}איור 2\\[2cm]}} \\[0.5cm]
\fbox{\parbox{5cm}{\centering\vspace{2cm}איור 3\\[2cm]}} &
\fbox{\parbox{5cm}{\centering\vspace{2cm}איור 4\\[2cm]}}
\end{tabular}
\caption{רשת איורים $2\times2$: \entoc{$2\times2$ Figure Grid}}
\label{fig:grid}
\end{figure}

איור \ref{fig:grid} מציג רשת של ארבעה איורים.

% ==================== COMPLEX FIGURES ====================

\hebrewsection{איורים מורכבים: \entoc{Complex Figures}}

\hebrewsubsection{דיאגרמה עם טקסט: \entoc{Diagram with Text}}

\begin{figure}[h]
\centering
\fbox{\parbox{12cm}{
    \centering
    \vspace{1cm}
    \textbf{ארכיטקטורת המערכת}\\[0.5cm]
    \begin{tabular}{c}
    \fbox{\parbox{8cm}{\centering שכבת קלט / \en{Input Layer}\\[0.3cm]}} \\[0.3cm]
    $\downarrow$ \\[0.3cm]
    \fbox{\parbox{8cm}{\centering שכבת עיבוד / \en{Processing Layer}\\[0.3cm]}} \\[0.3cm]
    $\downarrow$ \\[0.3cm]
    \fbox{\parbox{8cm}{\centering שכבת פלט / \en{Output Layer}\\[0.3cm]}}
    \end{tabular}
    \vspace{1cm}
}}
\caption{דיאגרמת ארכיטקטורה: \entoc{Architecture Diagram} - מבנה שלוש שכבות}
\label{fig:architecture}
\end{figure}

איור \ref{fig:architecture} מציג ארכיטקטורת מערכת בת שלוש שכבות.

\hebrewsubsection{גרפים ונתונים: \entoc{Graphs and Data}}

\begin{figure}[h]
\centering
\fbox{\parbox{14cm}{
    \centering
    \vspace{1cm}
    \textbf{תוצאות הניסוי}\\[0.5cm]
    \begin{tabular}{|c|c|c|c|}
    \hline
    זמן & ערך א & ערך ב & ערך ג \\
    \hline
    \hebyear{2020} & \num{10} & \num{20} & \num{30} \\
    \hebyear{2021} & \num{15} & \num{25} & \num{35} \\
    \hebyear{2022} & \num{20} & \num{30} & \num{40} \\
    \hebyear{2023} & \num{25} & \num{35} & \num{45} \\
    \hebyear{2024} & \num{30} & \num{40} & \num{50} \\
    \hebyear{2025} & \num{35} & \num{45} & \num{55} \\
    \hline
    \end{tabular}\\[0.5cm]
    \small{הנתונים מראים עלייה קבועה לאורך השנים}\\
    \vspace{0.5cm}
}}
\caption{נתוני ניסוי: \entoc{Experimental Data} - מגמות לאורך זמן}
\label{fig:data}
\end{figure}

הנתונים באיור \ref{fig:data} מציגים מגמת עלייה עקבית.

% ==================== CAPTIONS AND REFERENCES ====================

\hebrewsection{כיתובים והפניות: \entoc{Captions and References}}

\hebrewsubsection{כיתובים מפורטים: \entoc{Detailed Captions}}

\begin{figure}[h]
\centering
\fbox{\parbox{10cm}{
    \centering
    \vspace{3cm}
    איור עם כיתוב מפורט\\
    \vspace{3cm}
}}
\caption[כיתוב קצר: \entoc{Short Caption}]{כיתוב ארוך ומפורט: \entoc{Long Detailed Caption} - איור זה מדגים שימוש בכיתוב ארוך המסביר בפירוט את תוכן האיור. ניתן לכלול הסברים נוספים, הפניות למחקרים \cite{vaswani2017attention}, ואפילו נוסחאות מתמטיות כמו $y = mx + b$. הכיתוב הקצר מופיע ברשימת האיורים.}
\label{fig:detailed}
\end{figure}

\hebrewsubsection{הפניות מרובות: \entoc{Multiple References}}

במסמך זה הצגנו מגוון איורים:
\begin{itemize}
    \item איור בסיסי (איור \ref{fig:basic})
    \item איורים עם מיקומים שונים (איורים \ref{fig:top}, \ref{fig:bottom}, \ref{fig:page})
    \item איורים מרובים (איורים \ref{fig:sidebyside}, \ref{fig:grid})
    \item איורים מורכבים (איורים \ref{fig:architecture}, \ref{fig:data})
    \item איור עם כיתוב מפורט (איור \ref{fig:detailed})
\end{itemize}

% ==================== SUBFIGURES ====================

\hebrewsection{תת-איורים: \entoc{Subfigures}}

\begin{figure}[h]
\centering
\begin{tabular}{cc}
(א) & (ב) \\
\fbox{\parbox{5cm}{\centering\vspace{2cm}תת-איור א\\[2cm]}} &
\fbox{\parbox{5cm}{\centering\vspace{2cm}תת-איור ב\\[2cm]}} \\[0.5cm]
(ג) & (ד) \\
\fbox{\parbox{5cm}{\centering\vspace{2cm}תת-איור ג\\[2cm]}} &
\fbox{\parbox{5cm}{\centering\vspace{2cm}תת-איור ד\\[2cm]}}
\end{tabular}
\caption{איור עם תת-איורים: \entoc{Figure with Subfigures} - (א) חלק ראשון, (ב) חלק שני, (ג) חלק שלישי, (ד) חלק רביעי}
\label{fig:subfigures}
\end{figure}

איור \ref{fig:subfigures} מכיל ארבעה תת-איורים המסומנים באותיות עבריות.

% ==================== TIKZ FIGURES ====================

\hebrewsection{איורי TikZ: \entoc{TikZ Figures}}

\begin{figure}[h]
\centering
\begin{english}
\begin{tikzpicture}
    % Nodes
    \node[draw, circle] (A) at (0,0) {א};
    \node[draw, circle] (B) at (3,0) {ב};
    \node[draw, circle] (C) at (6,0) {ג};
    \node[draw, circle] (D) at (1.5,-2) {ד};
    \node[draw, circle] (E) at (4.5,-2) {ה};

    % Edges
    \draw[->] (A) -- (B);
    \draw[->] (B) -- (C);
    \draw[->] (A) -- (D);
    \draw[->] (B) -- (D);
    \draw[->] (B) -- (E);
    \draw[->] (C) -- (E);
    \draw[->] (D) -- (E);
\end{tikzpicture}
\end{english}
\caption{גרף מכוון: \entoc{Directed Graph} - דוגמה עם \en{TikZ}}
\label{fig:tikz}
\end{figure}

איור \ref{fig:tikz} נוצר באמצעות חבילת \en{TikZ}.

% ==================== BEST PRACTICES ====================

\hebrewsection{עקרונות מומלצים: \entoc{Best Practices}}

\hebrewsubsection{המלצות לשימוש באיורים: \entoc{Figure Usage Recommendations}}

\begin{enumerate}
    \item \textbf{גודל מתאים:} וודאו שהאיור קריא ולא גדול מדי
    \item \textbf{כיתובים ברורים:} השתמשו בכיתובים דו-לשוניים עם \texttt{\textbackslash entoc\{\}}
    \item \textbf{מיקום נכון:} בחרו במיקום המתאים - \texttt{[h]}, \texttt{[t]}, \texttt{[b]}, או \texttt{[p]}
    \item \textbf{הפניות:} השתמשו תמיד ב-\texttt{\textbackslash label} ו-\texttt{\textbackslash ref}
    \item \textbf{רזולוציה:} לתמונות אמיתיות, השתמשו ברזולוציה של לפחות 300 DPI
\end{enumerate}

\hebrewsubsection{פורמטים נתמכים: \entoc{Supported Formats}}

התבנית תומכת במגוון פורמטים:
\begin{itemize}
    \item \textbf{PDF:} מומלץ לגרפיקה וקטורית
    \item \textbf{PNG:} מומלץ לצילומי מסך ודיאגרמות
    \item \textbf{JPG:} מומלץ לתמונות צילום
    \item \textbf{TikZ:} ליצירת גרפיקה ישירות ב-\LaTeX
\end{itemize}

% ==================== SUMMARY ====================

\hebrewsection{סיכום: \entoc{Summary}}

מסמך זה הדגים:
\begin{itemize}
    \item שימוש בסביבת \texttt{figure} ובפקודת \texttt{\textbackslash hebrewfigure}
    \item אפשרויות מיקום שונות (\texttt{[h]}, \texttt{[t]}, \texttt{[b]}, \texttt{[p]})
    \item איורים מרובים וסידורים שונים
    \item כיתובים דו-לשוניים עם \texttt{\textbackslash entoc\{\}}
    \item הפניות צולבות לאיורים
    \item שילוב עם \en{TikZ} ליצירת גרפיקה
    \item עקרונות מומלצים לעבודה עם איורים
\end{itemize}

השימוש הנכון באיורים מעשיר את המסמך האקדמי ומקל על הבנת החומר.

% ==================== BIBLIOGRAPHY ====================

\newpage
\printhebrewbibliography
\printenglishbibliography

\end{document}