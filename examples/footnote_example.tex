% ================================================
% FOOTNOTE EXAMPLE - Hebrew Academic Template v5.0
% ================================================
% This example focuses on footnotes in Hebrew RTL documents
% Compiles to approximately 3-4 pages
% Copyright (c) 2025 Dr. Segal Yoram. All rights reserved.

\documentclass{hebrew-academic-template}

% Add bibliography file
\addbibresource{example_references.bib}

% Title page information
\hebrewtitle{דוגמת הערות שוליים - תבנית אקדמית עברית}
\englishtitle{Footnote Example - Hebrew Academic Template v5.0}
\hebrewauthor{ד"ר סגל יורם}
\date{\textenglish{November 2025}}

\begin{document}

\maketitle
\tableofcontents
\newpage

% ==================== INTRODUCTION ====================

\hebrewsection{מבוא להערות שוליים: \entoc{Introduction to Footnotes}}

מסמך זה מדגים שימוש מקיף בהערות שוליים בתבנית האקדמית העברית.
הערות שוליים הן כלי חשוב במסמכים אקדמיים\footnote{הערות שוליים מאפשרות להוסיף מידע נוסף מבלי לשבור את זרימת הטקסט הראשי.} ומשמשות למגוון מטרות.

% ==================== BASIC FOOTNOTES ====================

\hebrewsection{הערות שוליים בסיסיות: \entoc{Basic Footnotes}}

\hebrewsubsection{הערות פשוטות: \entoc{Simple Footnotes}}

טקסט עברי רגיל עם הערת שוליים פשוטה\footnote{זוהי הערת שוליים פשוטה בעברית.}.
ניתן להוסיף מספר הערות שוליים באותו משפט\footnote{הערה ראשונה.}\footnote{הערה שנייה.}.

הערות שוליים יכולות להכיל מגוון רחב של תוכן\footnote{כולל טקסט ארוך יותר שמסביר נקודה מסוימת בפירוט רב יותר. הערות כאלה שימושיות במיוחד כאשר רוצים להרחיב על נושא מבלי להכביד על הטקסט הראשי.}.

\hebrewsubsection{הערות עם תוכן מעורב: \entoc{Footnotes with Mixed Content}}

הערות שוליים יכולות לכלול טקסט מעורב\footnote{הערה עם עברית ו-\en{English text} יחד, כולל מספרים כמו \num{42} ואחוזים כמו \percent{95.5}.}.

ניתן גם לכלול נוסחאות מתמטיות בהערות\footnote{למשל, הנוסחה $E = mc^2$ או הביטוי $\sum_{i=1}^{n} i = \frac{n(n+1)}{2}$.}.

קוד יכול להופיע בהערות שוליים\footnote{לדוגמה: \code{print("Hello World")} או \en{def function():}.}.

% ==================== ADVANCED FOOTNOTES ====================

\hebrewsection{הערות שוליים מתקדמות: \entoc{Advanced Footnotes}}

\hebrewsubsection{הערות עם ציטוטים: \entoc{Footnotes with Citations}}

לעתים נדרש לצטט מקורות בתוך הערות שוליים\footnote{כפי שמוצג במחקר של \cite{vaswani2017attention}, טכנולוגיית \en{Transformer} חוללה מהפכה בתחום.}.

הערות יכולות להכיל ציטוטים מרובים\footnote{מחקרים רבים \cite{devlin2018bert,brown2020language} תומכים בגישה זו. ראה גם \cite[עמ' 42]{hebrew_nlp_2023} לדיון מפורט.}.

\hebrewsubsection{הערות מקוננות ומורכבות: \entoc{Nested and Complex Footnotes}}

הערות שוליים יכולות להכיל רשימות\footnote{%
הנקודות החשובות הן:
\begin{itemize}
    \item נקודה ראשונה
    \item נקודה שנייה עם \en{English}
    \item נקודה שלישית עם מספר \num{100}
\end{itemize}
שימו לב שכל הנקודות חשובות באותה מידה.}.

ניתן גם לכלול טבלאות קטנות בהערות\footnote{%
\begin{tabular}{|c|c|}
\hline
ערך א & \num{10} \\
\hline
ערך ב & \num{20} \\
\hline
\end{tabular}
- טבלה פשוטה להמחשה.}.

% ==================== FOOTNOTE NUMBERING ====================

\hebrewsection{מספור הערות שוליים: \entoc{Footnote Numbering}}

\hebrewsubsection{מספור רציף: \entoc{Sequential Numbering}}

הערות שוליים ממוספרות באופן רציף\footnote{זוהי ההערה מספר \thefootnote{} במסמך.}.
המספור ממשיך לאורך כל המסמך\footnote{וזוהי ההערה מספר \thefootnote{}.}.

ניתן להתייחס להערות קודמות\footnote{\label{fn:special}הערה מיוחדת שנרצה להפנות אליה.}.
כפי שראינו בהערה \ref{fn:special}, ניתן ליצור הפניות צולבות.

\hebrewsubsection{סוגי הערות שונים: \entoc{Different Types of Footnotes}}

הערות הסבר\footnote{הערת הסבר: מונח זה מתייחס ל...}.
הערות ביבליוגרפיות\footnote{למידע נוסף ראה: ספר א', פרק ב', עמוד ג'.}.
הערות מתודולוגיות\footnote{השיטה שנבחרה מבוססת על...}.

% ==================== ENGLISH SECTION WITH FOOTNOTES ====================

\englishsection{English Section with Footnotes}

\startenglish
This section demonstrates footnotes in English text\footnote{This is an English footnote with proper LTR alignment.}.

Footnotes can contain mixed content even in English sections\footnote{Including Hebrew text like \heb{טקסט עברי} and formulas like $x^2 + y^2 = r^2$.}.

Multiple footnotes work correctly\footnote{First English footnote.}\footnote{Second English footnote with \heb{עברית}.}.
\stopenglish

% ==================== FOOTNOTES IN SPECIAL CONTEXTS ====================

\hebrewsection{הערות בהקשרים מיוחדים: \entoc{Footnotes in Special Contexts}}

\hebrewsubsection{הערות בטבלאות: \entoc{Footnotes in Tables}}

\begin{hebrewtable}[h]
\caption{טבלה עם הערות שוליים: \entoc{Table with Footnotes}}
\begin{rtltabular}{|c|c|c|}
\hline
\mixedcell{\textbf{שם / \en{Name}}} & \mixedcell{\textbf{ערך / \en{Value}}} & \mixedcell{\textbf{הערה / \en{Note}}} \\
\hline
\mixedcell{פריט א\footnote{הערה לפריט א.}} & \num{100} & \mixedcell{רגיל} \\
\hline
\mixedcell{פריט ב} & \num{200}\footnote{ערך זה נמדד ב-\hebyear{2025}.} & \mixedcell{מיוחד} \\
\hline
\mixedcell{פריט ג} & \num{300} & \mixedcell{חשוב\footnote{ראה פירוט בנספח.}} \\
\hline
\end{rtltabular}
\end{hebrewtable}

\hebrewsubsection{הערות באיורים: \entoc{Footnotes in Figures}}

\begin{figure}[h]
\centering
\fbox{\parbox{10cm}{
    \centering
    \vspace{2cm}
    דיאגרמה עם הערה\footnote{הערת שוליים מתוך איור - מקרה מיוחד.}\\
    \en{Diagram with footnote}\\
    \vspace{2cm}
}}
\caption{איור עם הערת שוליים: \entoc{Figure with Footnote}}
\end{figure}

\hebrewsubsection{הערות בנוסחאות: \entoc{Footnotes in Equations}}

נוסחה מתמטית עם הערה:
\begin{equation}
E = mc^2 \footnote{המשוואה המפורסמת של איינשטיין.}
\end{equation}

או בתוך טקסט מתמטי: $x_{\text{max}}$\footnote{הערך המקסימלי של $x$.} = \num{100}.

% ==================== FOOTNOTE FORMATTING ====================

\hebrewsection{עיצוב הערות שוליים: \entoc{Footnote Formatting}}

\hebrewsubsection{הערות ארוכות: \entoc{Long Footnotes}}

טקסט עם הערה ארוכה במיוחד\footnote{%
זוהי הערת שוליים ארוכה במיוחד שמדגימה כיצד הערות ארוכות מתנהגות במסמך.
הערה זו כוללת מספר משפטים ונועדה להראות שהערות יכולות להכיל תוכן מפורט.
\par
ניתן אפילו לכלול פסקה חדשה בתוך הערת השוליים.
זה שימושי כאשר צריך להסביר נושא מורכב או להוסיף הקשר נרחב.
\par
הערות כאלה נפוצות במיוחד במסמכים אקדמיים מעמיקים, כאשר המחבר רוצה לספק מידע נוסף מבלי להפריע לזרימת הטקסט הראשי.
שימוש נכון בהערות שוליים יכול להעשיר משמעותית את המסמך.}.

\hebrewsubsection{סיכום השימוש בהערות: \entoc{Summary of Footnote Usage}}

הערות השוליים במסמך זה הדגימו\footnote{סיכום הדגמה בהערת שוליים אחרונה.}:

\begin{itemize}
    \item הערות פשוטות בעברית
    \item הערות עם תוכן מעורב עברית/אנגלית
    \item הערות עם מספרים ונוסחאות
    \item הערות עם ציטוטים ביבליוגרפיים
    \item הערות בטבלאות ואיורים
    \item הערות ארוכות ומורכבות
    \item מספור והפניות צולבות
\end{itemize}

% ==================== CONCLUSIONS ====================

\hebrewsection{סיכום: \entoc{Conclusions}}

מסמך זה הציג שימוש מקיף בהערות שוליים בתבנית האקדמית העברית.
הדגמנו מגוון רחב של שימושים והקשרים שונים להערות שוליים, והראינו שהתבנית תומכת בכל סוגי התוכן בהערות.

השימוש הנכון בהערות שוליים מאפשר:
\begin{enumerate}
    \item הוספת מידע משלים מבלי לשבור את זרימת הטקסט
    \item מתן הפניות והסברים נוספים
    \item ציון מקורות משניים
    \item הבהרות טכניות ומתודולוגיות
\end{enumerate}

% ==================== BIBLIOGRAPHY ====================

\newpage
\printhebrewbibliography
\printenglishbibliography

\end{document}