% ================================================
% BEGINNER EXAMPLE - Hebrew Academic Template v5.0
% ================================================
% This example demonstrates basic features for beginners
% Compiles to approximately 5-7 pages
% Copyright (c) 2025 Dr. Segal Yoram. All rights reserved.

\documentclass{hebrew-academic-template}

% Add bibliography file
\addbibresource{example_references.bib}

% Title page information
\hebrewtitle{דוגמה למתחילים - תבנית אקדמית עברית}
\englishtitle{Beginner Example - Hebrew Academic Template v5.0}
\hebrewauthor{ד"ר סגל יורם}
\hebrewversion{גרסה \textenglish{5.0}}
\date{\textenglish{November 2025}}

\begin{document}

\maketitle
\tableofcontents
\newpage

% ==================== INTRODUCTION ====================
% Demonstrates: \hebrewsection with \entoc{}, basic Hebrew/English mixing

\hebrewsection{מבוא: \entoc{Introduction}}

זהו מסמך דוגמה למתחילים המדגים את היכולות הבסיסיות של התבנית האקדמית העברית גרסה 5.0.
התבנית מאפשרת כתיבה בעברית עם שילוב טקסט באנגלית כמו \en{Machine Learning} ו-\en{Artificial Intelligence}.

% Demonstrate basic text direction commands
אנו יכולים להשתמש במספרים כמו \num{100}, \num{250.5}, ו-\num{1000000} בתוך הטקסט העברי.
השנה הנוכחית היא \hebyear{2025} והאחוזים מוצגים כך: \percent{95.5}.

% Basic inline math
ביטויים מתמטיים פשוטים כמו $E = mc^2$ או $\alpha + \beta = \gamma$ משתלבים בטבעיות בטקסט העברי.
מונחים טכניים באנגלית מוצגים באמצעות \ilm{inline math} או \en{English text}.

% ==================== BASIC LISTS ====================
% Demonstrates: itemize and enumerate environments

\hebrewsection{רשימות בסיסיות: \entoc{Basic Lists}}

\hebrewsubsection{רשימה עם תבליטים: \entoc{Bulleted List}}

רשימה פשוטה עם תבליטים:

\begin{itemize}
    \item פריט ראשון בעברית
    \item פריט שני עם מונח באנגלית: \en{Python}
    \item פריט שלישי עם מספר: \num{42}
    \item פריט רביעי עם אחוז: \percent{75}
\end{itemize}

\hebrewsubsection{רשימה ממוספרת: \entoc{Numbered List}}

רשימה ממוספרת עם שלבים:

\begin{enumerate}
    \item שלב ראשון: הכנת הנתונים
    \item שלב שני: עיבוד באמצעות \en{preprocessing}
    \item שלב שלישי: הרצת האלגוריתם
    \item שלב רביעי: ניתוח התוצאות
\end{enumerate}

% ==================== SIMPLE TABLE ====================
% Demonstrates: Basic table with \mixedcell

\hebrewsection{טבלה בסיסית: \entoc{Basic Table}}

להלן טבלה פשוטה המדגימה שימוש בפקודת \texttt{mixedcell} לתאים עם תוכן מעורב:

\begin{hebrewtable}[htbp]
\caption{נתונים בסיסיים: \entoc{Basic Data}}
\begin{rtltabular}{|c|c|c|}
\hline
\hebheader{\textbf{שם / \en{Name}}} & \hebheader{\textbf{ערך / \en{Value}}} & \hebheader{\textbf{יחידה / \en{Unit}}} \\
\hline
\hebcell{מהירות / \en{Speed}} & \num{100} & \en{km/h} \\
\hline
\hebcell{טמפרטורה / \en{Temperature}} & \num{25.5} & \en{°C} \\
\hline
\hebcell{דיוק / \en{Accuracy}} & \percent{98.7} & \hebcell{אחוזים / \en{Percent}} \\
\hline
\hebcell{זמן / \en{Time}} & \num{3.14} & \hebcell{שניות / \en{Seconds}} \\
\hline
\end{rtltabular}
\end{hebrewtable}

כפי שניתן לראות בטבלה, הפקודה \texttt{hebcell} מאפשרת שילוב נוח של עברית ואנגלית באותו תא.

% ==================== SIMPLE CODE ====================
% Demonstrates: Basic Python code box

\hebrewsection{קוד בסיסי: \entoc{Basic Code}}

להלן דוגמת קוד פשוטה ב-\en{Python}:

\begin{pythonbox}[\hebtitle{חישוב ממוצע: \en{Calculate Average}}]
# Simple function to calculate average
def calculate_average(numbers):
    """Calculate the average of a list of numbers"""
    if len(numbers) == 0:
        return 0
    total = sum(numbers)
    average = total / len(numbers)
    return average

# Example usage
data = [10, 20, 30, 40, 50]
result = calculate_average(data)
print(f"The average is: {result}")
\end{pythonbox}

הקוד מדגים פונקציה פשוטה לחישוב ממוצע של רשימת מספרים.

% ==================== CITATIONS ====================
% Demonstrates: Basic citations

\hebrewsection{ציטוטים בסיסיים: \entoc{Basic Citations}}

מחקרים מראים שאלגוריתמי \en{Deep Learning} משיגים תוצאות מרשימות \cite{vaswani2017attention}.
גישת \en{BERT} שפותחה על ידי \en{Google} הביאה לשיפור משמעותי בעיבוד שפה טבעית \cite{devlin2018bert}.

ניתן לצטט מספר מקורות יחד \cite{vaswani2017attention,devlin2018bert} או להוסיף הפניה לעמוד מסוים \cite[עמ' 42]{hebrew_nlp_2023}.

% ==================== BASIC FIGURE ====================
% Demonstrates: Simple figure with caption

\hebrewsection{איור בסיסי: \entoc{Basic Figure}}

\begin{figure}[htbp]
\centering
\includegraphics[width=0.7\textwidth]{images/example-figure1.png}
\caption{איור פשוט: \entoc{Simple Figure}}
\label{fig:simple}
\end{figure}

איור \ref{fig:simple} מדגים הוספת איור בסיסי עם כיתוב בעברית ואנגלית.

% ==================== SIMPLE MATH ====================
% Demonstrates: Basic mathematical expressions

\hebrewsection{ביטויים מתמטיים: \entoc{Mathematical Expressions}}

משוואה פשוטה במרכז:

\begin{equation}
    f(x) = ax^2 + bx + c
    \label{eq:quadratic}
\end{equation}

משוואה \ref{eq:quadratic} היא הנוסחה הריבועית הסטנדרטית.

ביטויים מתמטיים נוספים:
\begin{itemize}
    \item אינטגרל: $\int_0^1 x^2 dx = \frac{1}{3}$
    \item סכום: $\sum_{i=1}^{n} i = \frac{n(n+1)}{2}$
    \item גבול: $\lim_{x \to \infty} \frac{1}{x} = 0$
\end{itemize}

% ==================== SUBSECTIONS ====================
% Demonstrates: \hebrewsubsection usage

\hebrewsection{תת-סעיפים: \entoc{Subsections}}

\hebrewsubsection{תת-סעיף ראשון: \entoc{First Subsection}}

זהו תת-סעיף ראשון המדגים את השימוש בפקודת \texttt{hebrewsubsection}.
הפקודה יוצרת תת-סעיף עם מספור אוטומטי וכותרת דו-לשונית.

\hebrewsubsection{תת-סעיף שני: \entoc{Second Subsection}}

תת-סעיף נוסף עם תוכן שונה. שימו לב שהמספור נעשה באופן אוטומטי.
ניתן להוסיף כמה תת-סעיפים שרוצים תחת כל סעיף ראשי.

% ==================== ENGLISH SECTION ====================
% Demonstrates: \englishsection with \startenglish and \stopenglish

\englishsection{English Section}

\startenglish
This is a pure English section that demonstrates left-to-right text flow.
All content in this section is in English and follows standard English typography conventions.

The template supports both Hebrew and English sections seamlessly.
You can include:
\begin{itemize}
    \item Bulleted lists in English
    \item Mathematical expressions like $y = mx + b$
    \item Code snippets and technical terms
    \item References to figures and equations
\end{itemize}

This section uses the \texttt{startenglish} and \texttt{stopenglish} commands to ensure proper text direction.
\stopenglish

% ==================== SUMMARY ====================
% Demonstrates: Conclusion section

\hebrewsection{סיכום: \entoc{Summary}}

מסמך זה הדגים את היכולות הבסיסיות של התבנית האקדמית העברית גרסה 5.0:

\begin{itemize}
    \item כתיבה בעברית עם שילוב אנגלית באמצעות \texttt{\textbackslash en\{\}}
    \item שימוש במספרים (\texttt{\textbackslash num\{\}}), שנים (\texttt{\textbackslash hebyear\{\}}), ואחוזים (\texttt{\textbackslash percent\{\}})
    \item יצירת סעיפים ותת-סעיפים עם \texttt{\textbackslash entoc\{\}} לתוכן העניינים
    \item טבלאות עם תוכן מעורב באמצעות \texttt{\textbackslash mixedcell\{\}}
    \item הוספת קוד עם \texttt{pythonbox}
    \item ציטוטים ביבליוגרפיים
    \item איורים וביטויים מתמטיים
    \item סעיפים באנגלית עם \texttt{\textbackslash startenglish} ו-\texttt{\textbackslash stopenglish}
\end{itemize}

התבנית מספקת כלים נוחים ופשוטים לכתיבת מסמכים אקדמיים בעברית עם אינטגרציה מלאה של תוכן באנגלית.

% ==================== BIBLIOGRAPHY ====================
% Print bibliography sections

\newpage
\printhebrewbibliography
\printenglishbibliography

\end{document}