% ================================================
% Table Examples - hebrew-academic-template v7.0.4
% Demonstrates all table layouts and themes
% ================================================

\documentclass{hebrew-academic-template}

\hebrewtitle{דוגמאות טבלאות -- \en{Table Examples}}
\englishtitle{Table Examples - hebrew-academic-template v7.0.4}
\hebrewauthor{תבנית אקדמית עברית}
\hebrewversion{גרסה \en{7.0.4}}
\date{\en{January 2026}}

\begin{document}

\maketitle

\begin{center}
\textbf{מילות מפתח:}
טבלאות, \en{tabularray}, ערכות נושא, עברית-אנגלית, נוסחאות
\end{center}

\vspace{1em}
\hrule
\vspace{1em}

% ============================================================================
% Section 1: Available Table Themes
% ============================================================================
\hebrewsection{ערכות נושא זמינות לטבלאות}

התבנית מספקת ארבע ערכות נושא (\en{themes}) מובנות לטבלאות:

\begin{enumerate}
    \item \textbf{\en{fancy}} -- ברירת מחדל: כותרת כחולה, כל הגבולות, כותרות מודגשות
    \item \textbf{\en{minimal}} -- סגנון \en{booktabs}: ללא קווים אנכיים, קווים אופקיים בלבד
    \item \textbf{\en{striped}} -- צבעים מתחלפים: שורות אפורות לסירוגין, כותרת כחולה
    \item \textbf{\en{academic}} -- סגנון אקדמי: קווים עבים למעלה ולמטה, נקי
\end{enumerate}

% ============================================================================
% Section 2: Basic Fancy Table
% ============================================================================
\hebrewsection{טבלת \en{Fancy} בסיסית}

הטבלה הבסיסית עם ערכת נושא \en{fancy} (ברירת מחדל):

\begin{fancytable}{ccc}{טבלת אמת של שער \en{XOR}}
$x_1$ & $x_2$ & \en{XOR} \\
0 & 0 & 0 \\
0 & 1 & 1 \\
1 & 0 & 1 \\
1 & 1 & 0 \\
\end{fancytable}

\hebrewsubsection{תחביר}

\begin{english}
\begin{verbatim}
\begin{fancytable}{colspec}{caption}
Header1 & Header2 & Header3 \\
data & data & data \\
\end{fancytable}
\end{verbatim}
\end{english}

% ============================================================================
% Section 3: Mixed Hebrew/English Content
% ============================================================================
\hebrewsection{תוכן מעורב עברית-אנגלית}

טבלה עם תוכן מעורב בכותרות ובתאים:

\begin{fancytable}{Hcc}{השוואת אלגוריתמים -- \en{Algorithm Comparison}}
שם האלגוריתם & סיבוכיות זמן & סיבוכיות מקום \\
\en{Bubble Sort} מיון בועות & $O(n^2)$ & $O(1)$ \\
\en{Merge Sort} מיון מיזוג & $O(n \log n)$ & $O(n)$ \\
\en{Quick Sort} מיון מהיר & $O(n \log n)$ & $O(\log n)$ \\
\en{Heap Sort} מיון ערימה & $O(n \log n)$ & $O(1)$ \\
\end{fancytable}

% ============================================================================
% Section 4: Table with Formulas
% ============================================================================
\hebrewsection{טבלה עם נוסחאות מתמטיות}

טבלה המכילה נוסחאות בכותרות ובתאים:

\begin{fancytable}{Hcc}{פונקציות אקטיבציה -- \en{Activation Functions}}
שם הפונקציה & הנוסחה $f(x)$ & הנגזרת $f'(x)$ \\
\en{Sigmoid} סיגמואיד & $\frac{1}{1+e^{-x}}$ & $f(x)(1-f(x))$ \\
\en{Tanh} טנגנס היפרבולי & $\frac{e^x-e^{-x}}{e^x+e^{-x}}$ & $1-f(x)^2$ \\
\en{ReLU} & $\max(0,x)$ & $\begin{cases} 1 & x>0 \\ 0 & x \leq 0 \end{cases}$ \\
\en{Softmax} & $\frac{e^{x_i}}{\sum_j e^{x_j}}$ & מורכב \\
\end{fancytable}

% ============================================================================
% Section 5: Theme Comparison
% ============================================================================
\hebrewsection{השוואת ערכות נושא}

\hebrewsubsection{ערכת נושא \en{fancy} (ברירת מחדל)}

\begin{table}[H]
\centering
\caption{ערכת נושא \en{fancy}}
\begin{tblrfancy}{ccc}
כותרת א' & כותרת ב' & כותרת ג' \\
נתון \en{1} & נתון \en{2} & נתון \en{3} \\
נתון \en{4} & נתון \en{5} & נתון \en{6} \\
נתון \en{7} & נתון \en{8} & נתון \en{9} \\
\end{tblrfancy}
\end{table}

\hebrewsubsection{ערכת נושא \en{minimal}}

\begin{table}[H]
\centering
\caption{ערכת נושא \en{minimal} -- סגנון \en{booktabs}}
\begin{tblrminimal}{ccc}
כותרת א' & כותרת ב' & כותרת ג' \\
נתון \en{1} & נתון \en{2} & נתון \en{3} \\
נתון \en{4} & נתון \en{5} & נתון \en{6} \\
נתון \en{7} & נתון \en{8} & נתון \en{9} \\
\end{tblrminimal}
\end{table}

\hebrewsubsection{ערכת נושא \en{striped}}

\begin{table}[H]
\centering
\caption{ערכת נושא \en{striped} -- שורות מתחלפות}
\begin{tblrstriped}{ccc}
כותרת א' & כותרת ב' & כותרת ג' \\
נתון \en{1} & נתון \en{2} & נתון \en{3} \\
נתון \en{4} & נתון \en{5} & נתון \en{6} \\
נתון \en{7} & נתון \en{8} & נתון \en{9} \\
נתון \en{10} & נתון \en{11} & נתון \en{12} \\
\end{tblrstriped}
\end{table}

\hebrewsubsection{ערכת נושא \en{academic}}

\begin{table}[H]
\centering
\caption{ערכת נושא \en{academic} -- סגנון אקדמי}
\begin{tblracademic}{ccc}
כותרת א' & כותרת ב' & כותרת ג' \\
נתון \en{1} & נתון \en{2} & נתון \en{3} \\
נתון \en{4} & נתון \en{5} & נתון \en{6} \\
נתון \en{7} & נתון \en{8} & נתון \en{9} \\
\end{tblracademic}
\end{table}

% ============================================================================
% Section 6: Customization Options
% ============================================================================
\hebrewsection{אפשרויות התאמה אישית}

\hebrewsubsection{רשימת אפשרויות}

להלן האפשרויות הזמינות להתאמה אישית של טבלאות \en{tabularray}:

\begin{fancytable}{lR{8cm}}{אפשרויות התאמה אישית -- \en{Customization Options}}
אפשרות & תיאור \\
\en{colspec} & הגדרת עמודות: \en{c} (מרכז), \en{l} (שמאל), \en{r} (ימין), \en{p\{width\}} (רוחב קבוע) \\
\en{hlines} & הוספת כל הקווים האופקיים \\
\en{vlines} & הוספת כל הקווים האנכיים \\
\en{row\{N\}} & עיצוב שורה \en{N}: \en{bg=color}, \en{fg=color}, \en{font=\textbackslash bfseries} \\
\en{column\{N\}} & עיצוב עמודה \en{N} \\
\en{cell\{R\}\{C\}} & עיצוב תא ספציפי בשורה \en{R} ועמודה \en{C} \\
\en{rowsep} & ריווח אנכי בין שורות \\
\en{colsep} & ריווח אופקי בין עמודות \\
\en{hline\{N\}} & עיצוב קו אופקי \en{N} (עובי, צבע) \\
\en{vline\{N\}} & עיצוב קו אנכי \en{N} \\
\end{fancytable}

\hebrewsubsection{דוגמה לטבלה מותאמת אישית}

\begin{table}[H]
\centering
\caption{טבלה עם התאמה אישית}
\begin{tblr}{
  colspec = {ccc},
  row{1} = {bg=green!20, font=\bfseries},
  row{3} = {bg=yellow!20},
  cell{2}{2} = {bg=red!20},
  hlines,
  vlines,
  hline{1,Z} = {1.5pt, blue},
}
כותרת א' & כותרת ב' & כותרת ג' \\
נתון 1 & תא מודגש & נתון 3 \\
שורה מודגשת & נתון 5 & נתון 6 \\
נתון 7 & נתון 8 & נתון 9 \\
\end{tblr}
\end{table}

% ============================================================================
% Section 7: Wide Column Tables
% ============================================================================
\hebrewsection{טבלאות עם עמודות רחבות}

טבלה עם עמודות ברוחב קבוע לטקסט ארוך:

\begin{fancytable}{R{3cm}R{4cm}R{4cm}}{השוואת שיטות למידה -- \en{Learning Methods}}
שיטה & יתרונות & חסרונות \\
למידה מונחית \en{Supervised} & דיוק גבוה כאשר יש נתונים מתויגים & דורש כמות גדולה של נתונים מתויגים \\
למידה לא מונחית \en{Unsupervised} & לא דורש תיוג, מגלה מבנים נסתרים & קשה להעריך את הביצועים \\
למידת חיזוק \en{Reinforcement} & לומד מניסיון ללא פיקוח ישיר & דורש הרבה אינטראקציות, איטי \\
\end{fancytable}

% ============================================================================
% Section 8: Legacy Table Support
% ============================================================================
\hebrewsection{תמיכה בטבלאות קודמות}

התבנית תומכת גם בתחביר הטבלאות הישן (\en{hebrewtable + rtltabular}):

\begin{hebrewtable}[H]
\caption{טבלה בתחביר ישן (תאימות לאחור)}
\begin{rtltabular}{|c|c|c|}
\hline
\textbf{\hebcellc{כותרת א'}} & \textbf{\hebcellc{כותרת ב'}} & \textbf{\hebcellc{כותרת ג'}} \\
\hline
\hebcellc{נתון \en{1}} & \hebcellc{נתון \en{2}} & \hebcellc{נתון \en{3}} \\
\hline
\hebcellc{נתון \en{4}} & \hebcellc{נתון \en{5}} & \hebcellc{נתון \en{6}} \\
\hline
\end{rtltabular}
\end{hebrewtable}

\textbf{הערה:} מומלץ להשתמש ב-\en{fancytable} החדש לטבלאות חדשות. בטבלאות ישנות, יש לעטוף תוכן עברי ב-\en{\textbackslash hebcellc\{\}}.

% ============================================================================
% Section 9: Quick Reference
% ============================================================================
\hebrewsection{מדריך מהיר}

\hebrewsubsection{יצירת טבלה פשוטה}

\begin{english}
\begin{verbatim}
\begin{fancytable}{ccc}{Caption Text}
Header1 & Header2 & Header3 \\
data1 & data2 & data3 \\
\end{fancytable}
\end{verbatim}
\end{english}

\hebrewsubsection{שימוש בערכת נושא ספציפית}

\begin{english}
\begin{verbatim}
\begin{table}[H]
\centering
\caption{Caption}
\begin{tblrminimal}{ccc}  % or tblrfancy, tblrstriped, tblracademic
...
\end{tblrminimal}
\end{table}
\end{verbatim}
\end{english}

\hebrewsubsection{טבלה מותאמת אישית}

\begin{english}
\begin{verbatim}
\begin{tblr}{
  colspec = {ccc},
  row{1} = {bg=blue!15, font=\bfseries},
  hlines, vlines,
}
...
\end{tblr}
\end{verbatim}
\end{english}

\end{document}
